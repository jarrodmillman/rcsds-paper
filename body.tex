\section{Introduction}

Few neuroimaging analyses are computationally reproducible,\footnote{Following
\citet{buckheit1995wavelab}, we define an analysis as computationally
reproducible when someone other than the original author of an analysis can
take the author's data, code and instructions, and reproduce their analysis.}
even between researchers in a single lab.
The process of analysis is typically ad-hoc and informal; an analysis is often
the fruit of a considerable amount of trial and error, based in part on scripts
and data structures that the author inherited from past and current members of
the lab.
This \emph{process} is
(a) confusing, leading to unclear hypotheses and conclusions,
(b) error prone, leading to incorrect conclusions and greater confusion,
and (c) an impractical foundation on which to build reproducible analyses.

The trifecta of confusion, error, and lack of reproducibility are ubiquitous
problems of computation that programmers have been fighting since before ``programmer''
became a job title. There are widely accepted tools and processes to
control this trifecta. These tools include the Unix command line, version
control, high-level readable programming languages, code review, breaking code
into manageable functions and objects, writing tests for all code,  and
continuous automatic test execution.

Most researchers accept that learning these techniques is desirable, but
object that a neuroimager has a limited amount of time they can spend in
class.  This time should be spent learning an overview of neuroimaging
analysis, rather than tools and process for efficient development.

In the course we describe here, we tested our suspicion that we could
effectively teach \emph{both} the tools for efficient and reproducible work \emph{and}
the principles of neuroimaging, by putting a substantial collaborative project
at the heart of the course, and putting the tools into immediate practice.

At intake, our students that had little or no prior exposure to neuroimaging,
or to the computational tools and process we listed above.  We set them the
open-ended task of designing and executing a project which was either a
replication or an extension of a published neuroimaging analysis, built from
code they had written themselves.  We required the analysis to be easily
reproducible; the graders had to be able to figure out from a short text file
how to use a simple set of commands to
fetch and validate the data, run the analysis, and generate the final report, including the
figures.  As we report in the results, our students were remarkably
successful in achieving this goal.  We argue that this approach to
training in neuroimaging is likely to be far more fruitful than traditional
imaging courses in preparing our students to do efficient, valid, and
reproducible work.

%%general crisis in modern scientific research ;
%%large data / complex analysis / diverse technical skills needed .
%%Neuroimaging is experiencing these problems as well ;
%%examples ...
%%better practices and training needed...
%We have witnessed the problems with computational reproducibility in
%neuroimaging first-hand.
%In addition to large data and complex analysis,
%neuroimaging research is inherently collaborative and
%multidisciplinary.
%%collaboration, multidisciplinary ;
%%data and analysis sharing ;
%%large multisite data collection ;
%%increasing number of extra data sources (e.g., genomic) ;
%
%In this article, we describe a statistics course we taught to master and
%undergraduate students at UC Berkeley.
%With the above concerns in mind, we took a practical rather than academic
%approach to teaching reproducible and collaborative statistical analysis of
%neuroimaging data.
%Students trained to use a set of tools and processes through an open-ended,
%semester-long neuroimaging data analysis project.
%While students taking statistics courses and those taking psychology or
%neuroscience courses typically have different backgrounds, we found (see
%\S~\ref{discussion}) that there was little difference between the two in their
%abilities to master the technical material necessary for computational
%reproducibility.
%As a result, we taught the course much like we would teach it if it was
%offered through a psychology or neuroscience department.
%
%
%\fixme{Showing rather than motivating as a theme for the approach in the
%course (pitch in and do rather than explain first).
%Until they see it, they won't know they need it.
%Our course is a proof of concept for our approach ...}
%
%%collaboration is not only a matter of learning tools
%%it also requires training not just neuroimagers, but
%%also statisticians and computer scientists.
%%we hope to see more courses like this one taught to
%%statiticians, computer scientists, applied mathematicians,
%%and physicists as well as psychologists and neuroscientists.
%
%%in the statistics department... ;
%%while the course in general could have focused on any scientific problem domain,
%%given our expertise we chose to focus on neuroimaging ;
%%taught it much like we teach beginning neuroscientists.
%
%%There was no requirement for students to have a background in neuroscience.
%%The intention of focusing on neuroimaging data was to provide a concrete problem
%%domain that exemplifies the types of programming and statistical challenges
%%present in many modern statistical applications.
%%Conversely, neuroimaging students taking a course like this also
%%gain familiarity with the types of programming and statistical challenges
%%present in many modern statistical applications.
%%This means they can more easily incorporate new applications
%%in their work.
%
%We begin by explaining the background and rationale (\S~\ref{background}) of
%our approach to teaching students how to analyze neuroimaging data.
%In material and methods (\S~\ref{methods}), we describe the course format, student audience,
%and curriculum.
%Student projects are presented in results (\S~\ref{results}).
%Finally, we discuss (\S~\ref{discussion}) adapting the course to
%different student audiences, ...

\section{Background and Rationale}\label{background}

Between us, we have many decades experience of teaching neuroimaging analysis.
% Matthew 1997- = 20
% JB 1997ish- 20
Like most other teachers of imaging, we have taught traditional courses with a
mixture of lectures covering the broad general ideas of the analysis, usually
combined with some practical workshops using imaging software.

We also have a great deal of experience of giving practical support for
imaging analysis to graduate students and post docs.

Our experience has gradually brought us to realize that we have three major
problems in the form of traditional teaching.

The first is that our students often leave their education with a superficial
understanding of imaging methods.  They are familiar with the steps, and the
terms, but when we ask them to reason about their analysis, they struggle.

The second problem is that the idealized picture in the lectures bears a
slight resemblance to their everyday work, which involves: using a Unix
computer, often remotely; collating and manipulating data; writing and sharing
scripts in computing languages such as Matlab or the Unix shell-scripting
languages.  Analyses can and do become messy and complicated, leading to
confusion and error.  Most labs expect students to learn how to do this on the
job, and few labs use current computing tools and process to control 
complexity and error.  As a corollary, it is usually difficult for a
researcher to share their analysis with another researcher, even in their own
lab, and it is hard for the principal investigator (PI) to check the analysis
for errors or invalid assumptions.

The final problem is that we have made it hard for imagers to collaborate with
other technical fields such as engineering and statistics.  Imaging software
packages are highly specific to brain imaging, and present an interface to
common statistical procedures that our students do not recognize.  Many
students do not have much background in standard engineering tools such as
convolution, the Fourier transform or numerical optimization.  In traditional
teaching, we usually gloss over the details of these fields, which saves us
time, but leaves the students without the understanding or vocabulary to
explain their analysis problem to other scientists in a way that makes it easy
for them to engage.  The result may be that imaging research is more likely to
live in a neuroscience ghetto that is effectively walled off from
collaboration and insight by other scientists.

After years of trying to patch up the problems caused by our own teaching, we
came to the conclusion that the traditional approach is based on a fundamental
error of emphasis.

Functional imaging analysis touches on many theoretical and practical fields,
including anatomy, physiology, signal and image processing, linear algebra,
statistics, computing, coding and data management.  Teaching all these topics
in depth would take longer than most students or institutions have time for.
We need to choose what we will teach, and what we will leave for the students
to pick up later. We have to decide what is fundamental, and what is
peripheral.

The traditional approach takes the top level to be fundamental.  It aims to
give the big picture, leaving the students to fill in the details. The
assumption is that the student will either know or pick up some of the
mathematical and practical foundations later, with the big picture as a
framework to guide them in choosing what to learn.

Our experience is that this assumption is quite incorrect.  It is rare for
students to go on to learn these foundations, and common for researchers to
continue to work with a limited and superficial understanding.  This has a
major negative impact on the quality of their work; because we did not teach
them these foundations, they come to believe that they do not need to know
them.  The foundations appear to them to be the preserve of clever
mathematical and computing types; they have no hope of understanding them in
any depth.  They treat the analysis as something special and magic, where the
rules are determined by practice, because they cannot hope to reason about
them.

We propose an alternative approach.  Instead of
concentrating on the big picture, we emphasize the foundations on which the
analysis is built, and their implementation in practice, with code and
computing.  We were inspired by the famous epithet of Richard Feynman, found
written on his blackboard after his death: ``What I cannot create, I do not
understand.''\footnote{\url{http://archives.caltech.edu/pictures/1.10-29.jpg}}
We give the students enough understanding of how the methods work, that
they could implement their own crude versions of the major algorithms.  In
teaching them to work with code, we also teach them standard programming
techniques for reducing complexity and controlling error, so they can think
clearly while they are working.  We encourage them to own their own learning,
and explore their data, by devoting a large proportion of the course and
course grade to an open-ended group project that must be public and completely
replicable, down to the figures used in the
final report.

We use two examples to illustrate the contrast between the
traditional approach and our own.

\subsection{Programming}

Traditional courses do not teach skills for programming; the hope is that the
student will pick up the code that they need while they are working, or by
taking an unrelated general course on programming.  Many lab heads without
programming expertise argue that teaching students to use code is not
necessary, because scientists do not need to be programmers.  Unfortunately,
imaging researchers do need to use, write and share code.  Without training,
our experience suggests that scientists tend to be inefficient programmers;
they take a long time to write code; their code is sloppy and
poorly-structured with many errors.  They improve very slowly with experience.
In practice, imaging researchers {\em are} programmers; but if we do not train
them to learn, they will be incompetent programmers, and the most likely
outcome is they will continue to be incompetent programmers for the rest of
their careers.

For this reason, we introduced standard coding process from the beginning of
our course, including testing, code review, measuring test coverage, version
control, and continuous automated testing with different software versions.
All final project work used public online repositories for version control,
code review, automated code coverage and testing.  The projects gave the
students an experience of efficient practice in collaboration with code and
data that they could apply to their future work.

\subsection{General Linear Model}

The General Linear Model (GLM) is a widely-used term in neuroimaging, to refer
to the generalization of multiple regression to analysis of variance (ANOVA)
and other standard group comparisons.  Statisticians generally prefer the term
``linear regression model'' or, simply, ``linear model.''

We have been thinking about how to teach the (general) linear model for many
years
\citep{poline2012general}.\footnote{\url{http://imaging.mrc-cbu.cam.ac.uk/imaging/PrinciplesStatistics}}\footnote{\label{glm_intro}
\url{http://matthew-brett.github.io/teaching/glm_intro.html}}

Functional imaging packages present the linear model as a matrix formulation
of multiple regression.  Many students have not taken linear algebra, or if
they have, do not remember it well.  The traditional teaching approach is to
show how the matrix formulation relates to simple and multiple regression over
a few slides in an hour-long talk, and then move on to more advanced topics
such as temporal filtering or random effects.  Although the students may have
the impression that they have understood, we have found that most of them are
unable to answer simple questions about the matrix formulation after such a
talk, and that they rarely go on to improve their understanding in their later
education.

Our current teaching works through the linear model very slowly. We found that
we needed to do this for the students to see that the matrix formulation of
the linear model is just another way of expressing statistical methods that
they already know, such as regression, $t$-tests and ANOVA.  We start with
simple regression, showing how the regression can be expressed as the addition
of vectors and vector / scalar multiplication.  This leads to the matrix
formulation of simple regression, and thence to multiple regression.  We
introduce dummy indicator variables to express group membership and show how
these relate to group means.  We do all this using simple Python code to show
them how the mathematics works in practice, and give them sample code that
they can edit and run for themselves.\cref{glm_intro}

In contrast, we rely on the students to work through more advanced concepts
such as temporal filtering and random effects.

\section{Material and Methods}\label{methods}

During the fall semester of 2015, we co-taught \emph{Reproducible and Collaborative
Statistical Data Science},\footnote{\url{http://www.jarrodmillman.com/stat159-fall2015/}}
a new project-based course offered through the Department of Statistics at UC Berkeley.
The course was open to upper-level undergraduates (as STAT 159) as well as
first-year graduate students (as STAT 259).

The course entry requirements were Berkeley courses Stat 133 (Concepts in
Computing with Data), Stat 134 (Concepts of Probability), and Stat 135
(Concepts of Statistics).  Together these courses provide basic
undergraduate-level familiarity with probability, statistics, and statistical
computing in R.
Many students were from statistics and/or computer science; however, we also
had students from many disciplines including cognitive science, psychology, and
architecture.
During the 15-week long semester, students were provided three hours of class
per week in two 90-minute sessions, plus two hours of labwork.
Students were expected to work at least eight hours per week outside class.
Project reports were due two weeks after the last class.

\subsection{Course overview}

\begin{quote}
A project-based introduction to statistical data analysis. Through case
studies, computer laboratories, and a term project, students learn
practical techniques and tools for producing statistically sound and
appropriate, reproducible, and verifiable computational answers to
scientific questions. Course emphasizes version control, testing,
process automation, code review, and collaborative programming.
Software tools include Bash, Git, Python, and \LaTeX.

\hfill\emph{---from the course catalog description}
\end{quote}

\emph{Reproducible and Collaborative Statistical Data Science} is a
project-based course that introduces students to reproducible and collaborative
statistical research, applied to a real scientific question.
Our objective for the course was for students to
(a) be proficient at the Unix command line,
(b) be fluent in using version control with Git,
(c) be able to write documents in Markdown and \LaTeX,
(d) be familiar with scientific computing in Python,
(e) understand the computational and statistical issues involved with reproducibility, and
(f) be familiar with computational issues in modern statistical data
analysis through hands-on analysis of neuroimaging data.

Below we focus on the course project (\S~\ref{project}) and
how we taught neuroimaging data analysis (\S~\ref{analysis}).
First, we describe the tools and process we taught the students to use.
We conclude this section by briefly describing the course
lectures, labs, notes, homeworks, readings, and quizzes.

\subsubsection{Tools and process}

Our approach was to teach the students efficient reproducible practice with
the standard tools that experts use for this purpose
\citep{millman2014developing}. Our motivating assumptions were twofold.
First, our own experience has taught us that the power of these tools only
becomes clear when you use them to do substantial work in collaboration with
your peers.  Second, we have found that teaching ``easy'' tools that need less
initial learning has the paradoxical effect of making it harder for learners
to move on to the more powerful and efficient tools that they will need for
their daily work.

We applied this approach in our choice of general computing tools and
neuroimaging analysis software.   For example, rather than teaching the user
interface of a specific neuroimaging analysis package, we taught scientific
coding with Python and its associated scientific libraries
\citep{millman2011python, perez2011python}, and then showed the students how to
perform standard statistical procedures on imaging data, using these tools
\citep{millman2007analysis}.

% we taught students the scientific Python software stack
% \citep{millman2011python, perez2011python} and software development best
% software development practices for computational reproducibility \citep{millman2014developing}.
% Python is increasingly popular for scientific computing and data science
% including for neuroimaging analysis \citep{millman2007analysis}.

\blockpar{Unix, text files, and the command line.}
The Unix environment is the computational equivalent of the scientists'
workbench \citep{preeyanon2014reproducible}.  The Unix command line, and the
Bash shell in particular, provides mature, well-documented tools for building
and executing sequences of commands that are readable, repeatable, and can be
stored in text files as scripts for later execution.  Quoting
\cite{wilson2014best}---the Bash shell makes it easier to ``make the computer
repeat tasks'' and ``save recent commands in a file for re-use.''

The GUI interface of operating systems such as Windows and macOS can obscure
the tree structure of the filesystem, and this in turn makes it harder for GUI
users to think about the organization of data in hierarchy of directories and
files.  The command line tools make the file system hierarchy explicit, and
therefore, make it easier to reason about data organization.

Linux and macOS are native Unix platforms and provide Unix / Bash command line
tools from a standard install.  Windows 10 provides the Unix / Bash tools as
an optional operating system install.

\blockpar{Version control with Git.}
Version control is a fundamental tool for organizing and storing code and
other products of data analysis.
Distributed version control allows many people to work in parallel on the
same set of files.
The tools associated with distributed version control make it much easier
for collaborators to review each other's work and suggest changes.

Git is the distributed version control system that has become the standard in
open source scientific computing. It is widely used in industry.
There are automated installers for all major operating systems.
Web platforms such as GitHub, Bitbucket, and GitLab provide web interfaces
to version control to simplify procedures such as code review, automated
testing and issue tracking (see below).

\blockpar{Pull requests and peer review.}
Regular peer review is one of the most important ways of continuing to learn
to be an effective author of correct code and data analysis.  Git and its
various web platforms provide a powerful tool for peer review.

Github, like other Git hosting platforms, provides an interface for creating
``Issues''.  These can be reports of errors in the code or analysis that need
to be tracked, larger scale suggestions for changes, or items in a To Do list.
%Students also used the Github Issues interface to record project discussions
%and actions taken by the group members.

%We told the students that we would use the Pull Request and Issue discussions
%as evidence for their contributions to the project, and as data for their
%final grades.


\blockpar{Scientific computing with Python.}
Python is a general purpose programming language,
popular for teaching and prevalent in industry.
It is used widely in science, with particular strength in data science,
astronomy and genetics.
Its strength in science rests on a stack of popular scientific libraries, including
as NumPy (computing with arrays); Scipy (scientific algorithms
including optimization and image processing); Matplotlib (2D plotting);
Scikit-Learn (machine learning) and Pandas (data science).

\blockpar{Functions and unit tests.}
Functions are reusable blocks of code used to perform a specific task.
They usually manipulate some input argument(s) and return some output
result(s).
There are several advantages to organizing your code into functions rather
than simply writing stream of consciousness scripts.
First, functions allow you to ``hide'' the details of a discrete bit of
code from view.
This can be used to make the steps of your analysis easier to see.
Second, it encourages you to think of your program as a collection of sub-steps.
Third, it allows you to reuse code instead of rewriting or copying and editing
it.
Fourth, it enables you to keep your variable namespace clean.
Finally, it also allows you to test small parts of your program in isolation
from the rest.

unit tests, doctests


\blockpar{Continuous integration.}
Several services to build and test software projects on GitHub 
building on the unit tests ...
coverage reports...

\blockpar{Document processing and markup languages.}
Separating the content from the format of a document is useful.
A markup language allows for generation of content in plain text, while also
maintaining a set of special syntax to annotate the content with structural
information.
In this way, the plain-text or ``source'' files are human-readable and easy to
version control.
To produce the final document, the source files must undergo a build or render
step where the markup syntax (and any associated style files) are passed to a
tool that knows how to interpret and render the content.

\fixme{Document generation with \LaTeX and pandoc.}

\blockpar{Reproducible workflows with Make.}
The venerable Make system was originally written to automate the process
of compiling and linking programs.
However, when working in the Unix command line environment, you are constantly
generating files by performing a sequence of commands perhaps depending on
other files.
%While exploring running these commands by hand is common.
Makefiles are machine-readable text files enabling reproducible workflows.
A basic \texttt{Makefile} consists of ``rules'' of the following form:
\begin{verbatim}
            target: sourcefile(s)
                command(s)
\end{verbatim}

%Theoretically, one could write a single program for each data analysis.
%download data, validate it, perform EDA, run the analysis, and generate
%the report.
%In practice, scientists rarely operate in this manner.
%It is more common to write several smaller scripts or programs
%and then run them by hand at the commandline.
%This enables ...
%However, this is error prone and it can be easy to forget the specific
%invocations made when running scripts by hand.

\subsubsection{Lectures, labs, notes, homeworks, readings, and quizzes}

The weekly lectures and labs prepared students---through a
series of guided exercises---with the basic skills and
background needed for the group project.
While we demonstrated basic methods and tools during lecture and lab,
students were expected to do additional research and reading in the course
of working on the group project.
For instance, they may have needed to use a specialized analysis
method or a software package not covered in the lectures or labs.

\fixme{We gave one or two lectures on Unix, Git, Python, Testing, and NumPy.
Weekly labs focused on this software stack.
Students were referred to our course notes on
Bash,\footnote{\url{http://www.jarrodmillman.com/rcsds/standard/bash.html}}
Git,\footnote{\url{http://www.jarrodmillman.com/rcsds/standard/git-intro.html}}
Make,\footnote{\url{https://www.youtube.com/watch?v=-Cp3jBBHQBE}}
%\footnote{\url{https://github.com/berkeley-scf/tutorial-make-workflows}}
Python,\footnote{\url{http://www.jarrodmillman.com/rcsds/standard/python.html}}
scientific Python packages,\footnote{\url{http://www.scipy-lectures.org/}}
%and \LaTeX.\footnote{\url{https://www.youtube.com/watch?v=8khoelwmMwo}}
%\footnote{\url{http://github.com/berkeley-scf/tutorial-latex-intro}}
Course work made use of the toolstack.
All course work was version controlled with Git
and submitted via GitHub.
Finally, the course project extensively exercised the following tools
and practices.}

For the first three lectures, we briefly introduced students to the Unix
environment and started teaching them Git, a popular version control system.
The fourth lecture was a high-level introduction to neuroimaging and
functional magnetic resonance imaging (or fMRI), in particular.
Then we quickly introduced students to scientific computing with Python.
By the the ninth lecture, we started focusing on analyzing fMRI data
with Python.
The data analysis lectures continued to develop the students skills with
Unix and scientific Python software stack through practical problems,
which arose as we showed them how to analyze fMRI data.

%During the first 
%core software stack brief lectures beginning of course ;
%supplemented with labs and homeworks ;
%an occasional portion of class to address a new core topic ;
%most of the lectures were devoted to analzing neuroimaging data..

%learn what we need to know, when we need to know it; just-in-time learning 

%homeworks

During the first part of the semester, we assigned two homeworks.
Students had two weeks to work on each homework.
Homeworks were assigned and submitted using private GitHub repositories.
Typically, assignments were given as a collection of unimplemented
functions.
These functions were documented according to the NumPy documentation
standard\footnote{\url{https://github.com/numpy/numpy/blob/master/doc/HOWTO_DOCUMENT.rst.txt}}
and including unit tests.
Students implemented the functions according to the documentation
and ran the unit tests to check that their code at least
returned the correct results for the provided tests.
Assignments were grade using unit tests, which were not provided to
the students before they submitted their work.
The homework reinforced the material we taught during the
beginning of the course and mainly focused on scientific programming
in Python.

%GitHub, doctest/unit tests, functions, basic Python and scientific Python...

\begin{figure}
\centering
\begin{Verbatim}[commandchars=\\\{\}]
\PY{n}{dna} \PY{o}{=} \PY{l+s+s1}{\PYZsq{}}\PY{l+s+s1}{ATGATTTTTCCATCTTTAAGTGCGATACTGTTTTGT}\PY{l+s+s1}{\PYZsq{}}
\PY{n}{dna\PYZus{}bases} \PY{o}{=} \PY{p}{[}\PY{l+s+s1}{\PYZsq{}}\PY{l+s+s1}{A}\PY{l+s+s1}{\PYZsq{}}\PY{p}{,} \PY{l+s+s1}{\PYZsq{}}\PY{l+s+s1}{C}\PY{l+s+s1}{\PYZsq{}}\PY{p}{,} \PY{l+s+s1}{\PYZsq{}}\PY{l+s+s1}{G}\PY{l+s+s1}{\PYZsq{}}\PY{p}{,} \PY{l+s+s1}{\PYZsq{}}\PY{l+s+s1}{T}\PY{l+s+s1}{\PYZsq{}}\PY{p}{]}
\PY{n}{rna\PYZus{}bases} \PY{o}{=} \PY{p}{[}\PY{l+s+s1}{\PYZsq{}}\PY{l+s+s1}{A}\PY{l+s+s1}{\PYZsq{}}\PY{p}{,} \PY{l+s+s1}{\PYZsq{}}\PY{l+s+s1}{C}\PY{l+s+s1}{\PYZsq{}}\PY{p}{,} \PY{l+s+s1}{\PYZsq{}}\PY{l+s+s1}{G}\PY{l+s+s1}{\PYZsq{}}\PY{p}{,} \PY{l+s+s1}{\PYZsq{}}\PY{l+s+s1}{U}\PY{l+s+s1}{\PYZsq{}}\PY{p}{]}
\PY{n}{basecomplement} \PY{o}{=} \PY{p}{\PYZob{}}\PY{l+s+s1}{\PYZsq{}}\PY{l+s+s1}{A}\PY{l+s+s1}{\PYZsq{}}\PY{p}{:} \PY{l+s+s1}{\PYZsq{}}\PY{l+s+s1}{T}\PY{l+s+s1}{\PYZsq{}}\PY{p}{,} \PY{l+s+s1}{\PYZsq{}}\PY{l+s+s1}{C}\PY{l+s+s1}{\PYZsq{}}\PY{p}{:} \PY{l+s+s1}{\PYZsq{}}\PY{l+s+s1}{G}\PY{l+s+s1}{\PYZsq{}}\PY{p}{,} \PY{l+s+s1}{\PYZsq{}}\PY{l+s+s1}{T}\PY{l+s+s1}{\PYZsq{}}\PY{p}{:} \PY{l+s+s1}{\PYZsq{}}\PY{l+s+s1}{A}\PY{l+s+s1}{\PYZsq{}}\PY{p}{,} \PY{l+s+s1}{\PYZsq{}}\PY{l+s+s1}{G}\PY{l+s+s1}{\PYZsq{}}\PY{p}{:} \PY{l+s+s1}{\PYZsq{}}\PY{l+s+s1}{C}\PY{l+s+s1}{\PYZsq{}}\PY{p}{\PYZcb{}}


\PY{k}{def} \PY{n+nf}{is\PYZus{}dna}\PY{p}{(}\PY{n}{dna}\PY{p}{)}\PY{p}{:}
    \PY{l+s+sd}{\PYZdq{}\PYZdq{}\PYZdq{}}
\PY{l+s+sd}{    Checks whether a string is a DNA string.}

\PY{l+s+sd}{    Parameters}
\PY{l+s+sd}{    \PYZhy{}\PYZhy{}\PYZhy{}\PYZhy{}\PYZhy{}\PYZhy{}\PYZhy{}\PYZhy{}\PYZhy{}\PYZhy{}}
\PY{l+s+sd}{    dna : string}
\PY{l+s+sd}{        A string (i.e., you can assume you get a string)}

\PY{l+s+sd}{    Returns}
\PY{l+s+sd}{    \PYZhy{}\PYZhy{}\PYZhy{}\PYZhy{}\PYZhy{}\PYZhy{}\PYZhy{}}
\PY{l+s+sd}{    out : bool}
\PY{l+s+sd}{        Returns True, if dna is a valid DNA string (i.e,}
\PY{l+s+sd}{        a string composed of the letters \PYZsq{}A\PYZsq{}, \PYZsq{}C\PYZsq{}, \PYZsq{}G\PYZsq{}, or}
\PY{l+s+sd}{        \PYZsq{}T\PYZsq{} (and False otherwise).}

\PY{l+s+sd}{    Hint}
\PY{l+s+sd}{    \PYZhy{}\PYZhy{}\PYZhy{}\PYZhy{}}
\PY{l+s+sd}{    Use the builtin set function and set methods.}

\PY{l+s+sd}{    Examples}
\PY{l+s+sd}{    \PYZhy{}\PYZhy{}\PYZhy{}\PYZhy{}\PYZhy{}\PYZhy{}\PYZhy{}\PYZhy{}}
\PY{l+s+sd}{    \PYZgt{}\PYZgt{}\PYZgt{} is\PYZus{}dna(\PYZsq{}ATGATT\PYZsq{})}
\PY{l+s+sd}{    True}
\PY{l+s+sd}{    \PYZgt{}\PYZgt{}\PYZgt{} is\PYZus{}dna(\PYZsq{}ATGATUC\PYZsq{})}
\PY{l+s+sd}{    False}
\PY{l+s+sd}{    \PYZgt{}\PYZgt{}\PYZgt{} is\PYZus{}dna(\PYZsq{}atgattZss1\PYZsq{})}
\PY{l+s+sd}{    False}
\PY{l+s+sd}{    \PYZgt{}\PYZgt{}\PYZgt{} is\PYZus{}dna(\PYZsq{}CAT\PYZsq{})}
\PY{l+s+sd}{    True}
\PY{l+s+sd}{    \PYZdq{}\PYZdq{}\PYZdq{}}
    \PY{k}{return} \PY{n+nb+bp}{NotImplemented}
\end{Verbatim}

\caption{Above we list the first few lines of \texttt{dna.py}.
The assignment was to edit this file by implementing the functions.
Including \texttt{is\_dna}, there were 9 functions
to implement in \texttt{dna.py}.
The first homework consisted of this Python file as well as two others.
It also included a \texttt{README} file, which included instructions
such as:}\label{fig:dna}
\begin{quotation}
The functions you are to implement are already defined in the \texttt{dna.py} and
\texttt{cipher.py} files respectively. The string that immediately follows the function
definition is called the \emph{docstring}. (The function definition is the line
starting with \texttt{def}). In this case, you are to use the description given in the
docstring as a guide for how to implement the function. At the end of each
docstring, there is an \texttt{Examples} section that contains lines that start with
\verb|>>>|. These are called \emph{doctests} as they are \emph{tests} for the code that are
contained in the docstring. If you run the doctests on the assignment before
doing any work, you'll notice that all of the tests fail. That's because none
of the functions have been implemented yet! Once you have implemented a
function, you can run the doctests again to make sure that function passes all
of its doctests.
\end{quotation}
\end{figure}

The first homework focused on basic Python and consisted of three parts.
The first two parts, \texttt{dna.py} and \texttt{cipher.py}
were intended to get students used to reading and writing Python code.
For \texttt{dna.py} students had to implement a series of functions and for
\texttt{cipher.py} they implemented a class.
%The functions and methods you are to implement are already defined in the \texttt{dna.py} and
%\texttt{cipher.py} files respectively.
%The functions and methods were documented, but not implemented.
%We also provided simple unit tests, which they could use to partially test their work.
See the code listing in Figure~\ref{fig:dna} for an example.
For the third part, students analyzed the State of the Union
speeches from 1790 to 2015.
This involved reading text files, data cleaning and preprocessing, as well
as implementing functions to represent a term document matrix as a list of
lists.
%To do this you will need to prepare the speeches in
%a form that is suitable for statistical analysis. 
%https://github.com/berkeley-stat159/assignments-fall2015/blob/master/hw1/sou/sou.md

The second homework focused on data analysis and consisted of two parts. 
%https://github.com/berkeley-stat159/assignments-fall2015/blob/master/hw2/num.py
For the first part, students implemented some of the functionality of NumPy's arrays.
Their implementations represented multidimensional arrays using a (linear) 1-D list
of numbers and a shape tuple.
We also required that their implementations handle arrays using either
row-major and column-major ordering.
While we did not ask them to implement all of NumPy's functionality,
we asked them to implement several functions including basic operations
such as indexing and reshaping as well as more complicated
reduction operations such as summing array elements over a given axis. 
% https://github.com/berkeley-stat159/assignments-fall2015/blob/master/hw2/diagnostics/diagnostics.md 
The second part focused on detecting outlier 3D volumes in a 4D fMRI image.
This included
(a) implementing functions on image arrays using NumPy,
(b) exploring fMRI data for outliers,
(c) running least-squares fits on different models, and
(d) making and saving plots with matplotlib.

Over the course of the semester, seven readings were assigned on a roughly
bi-weekly basis.
The readings consisted of articles that emphasized the core concepts
of the class, either with respect to scientific computing, or neuroimaging.
For the ``assignment'' portion of the reading, students were asked to compose
a two-paragraph write-up that both summarized the main points of the article,
and commented on it from their personal perspective.

In principle, the labs and quizzes were designed to emphasize hands-on 
experience with the computing tools and best-practices associated with
reproducible and collaborative computing (e.g. version control, \LaTeX, etc.).
The quizzes were held at the beginning of lab sessions, and heavily emphasized
the computing aspects of the course material as opposed to the statistical and
neuroimaging components covered in the lectures.
Quizzes were multiple choice and assigned and submitted via GitHub.
The remainder of the lab was devoted to providing hands-on experience via
collaborative work on breakout exercises.
These breakout exercises were formulated as small ``projects'' that were to be
attacked by groups of students with 3 to 4 students per team.
Prior to the breakout session, a small review of material relevant for the 
exercises was presented, often in the form of an interactive Jupyter notebook.

\subsection{Course project}\label{project}

The course centered around a semester-long group project.
Students worked in teams on their projects for three months.
%Students formed teams at the start of week 5 and final reports were submitted
%at the end of week 17.
The reason the project lasted for so long and was such a large portion of the grade (55\%)
is that we believe computational reproducibility and collaboration
are too abstract in the absence of a significant concrete project.
If a project lasts just a few weeks, it is easy to remember all the steps
without carefully tracking them.
If you have a tiny dataset and a small handful of tiny functions, you can just
throw them all in a directory and post them online.
Problems compound as data grows and analysis complexity increases as lines of
code multiply.

%\begin{quote}
%\begin{flushleft}
%Form teams \dotfill Week 5\\
%Project proposal \dotfill Week 6\\
%Progress presentation \& draft report \dotfill Week 12\\
%Project presentation \dotfill Week 15\\
%Final report \dotfill Week 17\\
%\end{flushleft}
%\end{quote}

Early in the semester (week~5) groups of three to five students formed teams
and immediately (week~6) submitted proposals to investigate
a published result using the analysis of neuroimaging data.
All teams chose a paper and accompanying dataset from
OpenfMRI,\footnote{\url{https://www.openfmri.org/}} a publicly-available
depository of MRI and EEG datasets
\citep{poldrack2013toward,poldrack2015openfmri}.
%Groups of three to five students formed teams.
%Each team chose a published neuroimaging study to investigate and
%decided what they meant by \emph{investigate} in consultation with us.
Each team's investigation involved
acquiring, cleaning, and curating data;
formulating scientific questions statistically;
developing appropriate statistical methods
to analyze the data to answer the scientific questions;
implementing those methods in robust, testable, reusable, extensible software;
applying the methods;
visualizing the results;
interpreting the results;
and communicating the results to others.
Every aspect of the project work was done in a computationally reproducible
and collaborative way.

\begin{figure}
\centering
\begin{minipage}[b]{0.4\textwidth}
\dirtree{%
.1 /.
.2 proposal.
.3 Makefile.
.3 proposal.bib.
.3 proposal.tex.
.2 LICENSE.
.2 README.md.
}
\subcaption{Initial repository template.}\label{fig:init-repo}
\end{minipage}
\begin{minipage}[b]{0.4\textwidth}
\dirtree{%
.1 /.
.2 proposal.
.3 Makefile.
.3 proposal.bib.
.3 proposal.tex.
.2 code.
.2 data.
.3 \_\_init\_\_.py.
.3 data.py.
.3 hashes.json.
.3 test\_data.py.
.2 report.
.3 Makefile.
.3 report.bib.
.3 report.tex.
.2 slides.
.3 Makefile.
.3 final.md.
.3 progress.md.
.2 .coveragerc.
.2 .travis.yml.
.2 requirements.txt.
.2 LICENSE.
.2 Makefile.
.2 README.md.
}
\subcaption{Final project template.}\label{fig:final-repo}
\end{minipage}

\caption{General caption ...}
\label{fig:repo}
\end{figure}

\blockpar{Repository.}
Once teams were formed, we created a publicly visible project repository for
each team with a \LaTeX template for their project proposals.
See Figure~\ref{fig:repo} for a skeleton project repository listing.

All project artifacts were stored in this repository.
We required students to use GitHub's pull request and review mechanism
for any work contributed to the project and asked them to try to have
three team members comment on each pull request.
During the first few lectures, we covered this workflow in depth
and made clear that the project grade would be based on
(a) the final report,
(b) the analysis code,
(c) whether we could run their analysis code to reproduce all
their results and figures, and
(d) how effectively the team collaborated using the techniques
taught in the course (e.g., pull requests, code review, testing).
%https://github.com/berkeley-stat159/project-template
Moreover, students were instructed to create a pull request
or issue with code and text and \texttt{@mention} one or more of us,
if they had any questions about their projects.

\blockpar{Proposal.}
Once we created repositories for each team, they had a week to propose what
they would do for their projects.
The project proposals involved
(a) identifying a published fMRI paper and the accompanying data,
(b) explaining the basic idea of the paper in a paragraph,
(c) confirming that they could download the data and that what they
downloaded passed basic sanity checks (e.g., correct number of subjects), and
(d) explaining what approach they intended to take for exploring
the data and paper.

We read and discussed their project proposals and then meet with each
team to help them refine their initial ideas.
Most teams proposals were initially too ambitious, so our feedback
was mostly around helping the students come up with more manageable
goals.
 
\blockpar{Data.}
Rather than having students commit raw data to their project repository, we had
them commit code to download and validate all raw data used in their project.
Write scripts ...
For example, their project \texttt{README} might tell us that \texttt{make data}
downloads all the relevant files for their study and \texttt{make validate}
verifies the content of the downloaded files.

\blockpar{Code.}
Most code was written as a collection of functions with tests with short
scripts that called these functions to perform the project data analysis.

All code submissions were added to the project via GitHub pull requests (PRs).
While we allowed students to decide when to merge PRs, we recommended that
all PRs should have at least three participants, coverage should not decrease,
and tests should pass before being merged.

Pull requests, code review, continuous integration, coverage reporting

\blockpar{Slides.}
Slides were written in Markdown.
These plain text files were committed to the project repository.
For example, if \texttt{progress.md} is the source file for a progress presentation,
then the \texttt{Makefile} might have the following recipe for building a nicely
formatted PDF file \texttt{progress.pdf}:
\begin{verbatim}
progress.pdf: progress.md
        pandoc -t beamer -s progress.md -o progress.pdf
\end{verbatim}

In addition to using beamer,
Pandoc\footnote{\url{http://pandoc.org/MANUAL.html\#producing-slide-shows-with-pandoc}}
can also output using other formats like reveal.js or slidy.

\blockpar{Report.}
We committed the following three files to each project repository:
\begin{itemize}
\item \texttt{paper/report.tex}
\item \texttt{paper/report.bib}
\item \texttt{Makefile}
\end{itemize}


\blockpar{\texttt{README}.}
Each project was required to put a \texttt{README} file at the top-level
of their repository to explain how to install, configure,
and run all aspects of their project.
Each \texttt{README} clearly stated  how to operate the
components of the project and in what order to do so.
In particular, they were required to include a section with
an enumerated list of task to carry out, like:
\begin{verbatim}
    make data - downloads the data 
    make validate - ensure the data is not corrupted
    make eda - generate figures from exploratory analysis
    make analysis - generate figures and results
    make report - build final report
\end{verbatim}

The \texttt{README} also contained other pertinent information:
for example, if the \texttt{make data} step downloads 5 Gb of data, it should
warn users of this so they make sure they have enough disk space and are on a
stable internet connection before trying to do so.

\subsection{Neuroimaging data analysis}\label{analysis}

When teaching neuroimaging data analysis, our aim was for students to
(a) understand the basic concepts in neuroimaging,
and how they relate to the wider world of statistics, engineering, computer science;
(b) be comfortable working with neuroimaging data and code, so they can write
their own basic algorithms, and understand other people's code;
(c) work with code and data in a way that will save them time, help them collaborate,
and continue learning.%this isn't quite parallel

For this course we concentrated on the statistical analysis of functional MRI
data.  We scheduled the teaching to give the students a sound background in
standard neuroimaging statistical analysis with a linear model.  We covered
the following topics:

\begin{itemize}

\item The NIfTI image format as a simple image container.
\item Four dimensional (4D) images; extracting voxel time-courses from 4D
    images.
\item The neuronal time-course model.
\item The hemodynamic time-course model, the hemodynamic response function and
    convolution;
\item Spatial smoothing with Gaussian kernel;
\item Simple regression, vector formulation, matrix formulation; least-squares
    cost-function; solving with the pseudoinverse; extension to multiple
        regression; dummy variables and matrix formulation of analysis of
        variance.
\item High-resolution sampling of regressors.
\item Parametric modulation of regressors.

% An example lesson illustrating the approach
% An exercise as illustration.
% Kind-of flipped.
% Centered on the project.

\end{itemize}

\section{Results}\label{results}

There were a total of eleven different research teams composed of three
to five students, each responsible for completing a final project of their
own design using datasets available through the openfMRI organization.
The scope of the projects were defined iteratively: each team submitted a 
proposal a third of the way through the semester, with the instructors providing
prompt feedback to help clarify the motivation and goals of each project.
This feedback process continued throughout the semester, with students 
submitting drafts according to various milestones defined within the project
timeline.
Feedback was not only provided to the students from the instructors: the
project evaluations included peer-review, where each research team was
responsible for cloning another team's project, running the tests and analyses,
and evaluating the resulting technical document.
The peer-review process proved particularly valuable, as the students 
benefited greatly from the exposure to the coding techniques and 
organizational principles of their peers.

... an advantage of having all students working on their own work

\begin{table}
\centering
\begin{tiny}
\begin{tabular}{|>{\bf}c|>{\raggedright}p{4cm}|>{\raggedright}p{4cm}|>{\raggedright\arraybackslash}p{4cm}|}
\hline
 & 
\multicolumn{1}{c|}{\textbf{\ding{51}-}}
 & 
\multicolumn{1}{c|}{\textbf{\ding{51}}}
 & 
\multicolumn{1}{c|}{\textbf{\ding{51}+}}
 \\
\hline

Questions
 & 
Questions overly simplistic, unrelated, or unmotivated
 & 
Questions appropriate, coherent, and motivated
 & 
Questions well motivated, interesting, insightful, and novel
 \\
\hline

Analysis
 & 
Choice of analysis overly simplistic or incomplete
 & 
Analysis appropriate
 & 
Analysis appropriate, complete, advanced, and informative
 \\
\hline

Results
 & 
Conclusions missing, incorrect, or not based on analysis

Inappropriate choice of plots; poorly labeled plots; plots missing
 & 
Conclusions relevant, but partially correct or partially complete

Plots convey information but lack context for interpretation
 & 
Relevant conclusions tied to analysis and context

Plots convey information correctly with adequate and appropriate reference information
 \\
\hline

Collaboration
 & 
Few members contributed substantial effort or each members worked on only part of project
 & 
All members contributed substantial effort and everyone contributed to all aspects of project
 & 
All members contributed substantial effort to each project aspect
 \\
\hline

Tests
 & 
Tests incomplete, incorrect, or missing
 & 
Tests cover most of the project code
 & 
Extensive and comprehensive testing
 \\
\hline

Code review
 & 
Pull requests not adequately used, reviewed, or improved
 & 
Pull requests adequately used, reviewed, and improved
 & 
Code review substantial and extensive
 \\
\hline

Documentation
 & 
Poorly documented
 & 
Adequately documented
 & 
Well documented
 \\
\hline

Readability
 & 
Code readability inconsistent or poor
 & 
Code readability consistent and good quality
 & 
Code readability excellent
 \\
\hline

Organization
 & 
Poorly organized and structured repository
 & 
Reasonably organized and clear structure
 & 
Elegant and transparent code organization
 \\
\hline

Presentation
 & 
Verbal presentation illogical, incorrect, or incoherent

Visual presentation cluttered, disjoint, or illegible

Verbal and visual presentation unrelated
 & 
Verbal presentation partially correct but incomplete or unconvincing

Visual presentation is readable and clear

Verbal and visual presentation related
 & 
Verbal presentation correct, complete, and convincing

Visual presentation appealing, informative, and crisp

Verbal and visual presentation clearly related
 \\
\hline

Writing
 & 
Explanation illogical, incorrect, or incoherent
 & 
Explanation correct, complete, and convincing
 & 
Explanation correct, complete, convincing, and elegant
 \\
\hline

Reproduciblity
 & 
Code didn't run
 & 
Makefile recipes fetch data, validates fetched data, generates all results and figures in report
 & 
Makefiles generate EDA work and supplementary analysis
 \\
\hline

\end{tabular}

\caption{Project grading rubric.
An ``A'' was roughly two or more check pluses and no check minuses.}
\label{tab:rubric}
\end{tiny}
\end{table}

The evaluation of the final projects was based on criteria that emphasized the
underlying principles of reproducibility, collaboration, and technical
quality.
See Table~\ref{tab:rubric} for the final project grading rubric.
From the perspective of reproducibility, projects were evaluated on whether the
presented results could be generated from their repositories according to the
documentation they provided.
The code tests were also evaluated with respect to code coverage
and thoroughness.
The collaborative aspects of the project were evaluated using the information
from the project history provided by the version control, including
individual contributions and reviewing pull requests.
Finally, the technical quality of the project was evaluated both in terms of
the clarity of the final report and with respect to how well the proposed
goals of the study were met by the final results of the analysis.

There was a large disparity in the backgrounds of the students that constituted 
each research team, and this was reflected by a range of scopes for the
final projects.
The primary focus of the majority of projects was reproducing results
of published research from the openfMRI datasets using the neuroimaging 
analysis methods introduced in the course.
For example, several teams chose to focus on the open dataset provided by a
study entitled \textit{The neural basis of loss aversion in decision-making
under risk}\cite{tom2007neutral}.
The research teams implemented analyses based on the GLM as well extending
the analysis to include other regression techniques, comparing their results
to those of the published study.
Projects of this nature accounted for about 8 of the 11 research teams, and
on the whole were largely successful in demonstrating a practical understanding
of the relevant statistical techniques.
However, several teams expressed interest in using the open data sets to 
perform analyses beyond replicating the published results.
The investigation of modern machine learning techniques was of particular 
interest to several groups that included students with a stronger computer
science background.
These projects utilized supervised learning techniques in an attempt to 
develop predictive models based on input fMRI images.
The ability of students to apply these concepts (which were largely external
to the course) within the given reproducible and collaborative 
framework was impressive and, according to feedback from the students, quite
motivating and rewarding for them as well.
In all cases, the quality of the organization and emphasis on reproducibility
of the final projects was quite satisfactory, especially given the fact that
most of the students had no practical experience with the tools (Python, Git,
etc.) prior to this course.
Most teams demonstrated an understanding of the neuroimaging methods introduced
in the course by implementing analyses to replicate published results; and 
several went beyond this to expand their investigations to novel applications
of the open data.

%\url{https://raw.githubusercontent.com/jarrodmillman/rcsds/master/notes/rubric.rst}
%\url{http://www.jarrodmillman.com/rcsds/notes/rubric.pdf}

\section{Discussion}\label{discussion}

\subsection{Two roads to reproducible}

Most neuroimaging researchers agree that computational reproducibility is
desirable, but rare.  How should we adapt our teaching to make reproducibility
more common?

We believe the most common answer to this question, is that we should train
neuroimaging and other computational researchers as we do now, but add short
courses or bootcamps that teach a subset of the tools we taught here.  That
is, reproducibility is an addition on top of current training.

We believe this approach is doomed to failure, because the students are
building on an insecure foundation.  Once you have learned to work in an
informal and undisciplined way, it is difficult to switch to a process that
initially demands much more effort and thought.  Rigor and clarity are hard to
retrofit. To invert a common saying: "A week in the lab saves an hour in the
library".

For these reasons, our course took the opposite to the common approach. We
started with the tools and process for working with numerical data, and for
building their own analyses. We used this framework as a foundation on which
to build their understanding of the underlying ideas.  As we taught these
tools, we integrated them into their exercises, and made it clear how they
related to their work on the project.

Our claim is that this made our teaching and our students much more efficient.
The secure foundation made it easier for them to work with us and with each
other. As they started their project work, early in the course, they could
already see the value of these tools for clarity and collaboration. Our
students graduated from the course with significant experience of using the
same tools that experts use for sound and reproducible research.

\subsection{Did we really teach neuroimaging analysis?}

By design, our course covered tools and process as well as neuroimaging.  We
used neuroimaging as the example scientific application of these tools.  Can
we claim to have taught a substantial amount of neuroimaging in this class?

Class content specific to neuroimaging was a guest lecture in class 4 (of 25),
and teaching on the general linear model and principal component analysis from
classes 9 through 15.  We only attempted to cover the standard statistical
techniques needed for a basic analysis of a single run of FMRI data.  This is
a much narrower range than a standard neuroimaging course.  However, we argue
that we covered these topics in much greater depth that is typical for a
neuroimaging course, and that this formed the foundation for final projects
that were substantial and well-grounded.

As we argued in the introduction, typical imaging courses do not attempt to
teach the fundamental ideas of linear models, but assume this understanding
and move onto imaging specifics.  This assumption is almost entirely false for
neuroscience and psychology students, and mostly false for our own
students, most of whom already had some training from an undergraduate
statistics major. As a result of this incorrect assumption, it is rare for
students of neuroimaging to understand the statistics they are using.  We
taught both the linear model and principal components analysis from the first
principles of vector algebra, using the tools they had just learned to build a
simple analysis from basic components.  As a result, when the students got to
their projects, they had the tools they needed to build their own analysis
code for the neuroimaging data, both demonstrating and advancing their own
understanding.

We also note the difficulty of the task that we gave the students, and the
extent of their success.  We made clear that their project was an open-ended
exploration of an FMRI dataset and paper.  Very few students had any
experience or knowledge of FMRI before the course. The only guidance we gave
was that they should prefer well-curated datasets from the OpenFMRI
depository.  In order to design and implement their project, they had to
understand at least one published FMRI paper in some depth, with little
assistance from their instructors.  We gave no example projects for them to
review, or templates for them to follow.  As we have already described in the
results, the submitted projects were all substantial efforts to reproduce and
/ or extend the published results, and all included analysis code that they
had written themselves.

We did not teach the full range of neuroimaging analysis.  For example, we did
not cover pre-processing of the data, random effects analysis, or inference
with multiple comparisons.   Our claim is that what we did teach was a sound
and sufficient foundation from which they could build code to implement their
own analyses.  We argue that this is a level of competence that few
neuroimagers achieve even after considerable practical experience.

\subsection{Did we teach the right tools?}

We do not believe that the individual tools we chose were controversial. We
consciously taught the tools that we use ourselves, and that we would teach to
students working with us.

We could have used R instead of Python, but Python is a better language for
teaching, and has seen more development for neuroimaging and machine learning.

A more interesting question is whether we went too far in forcing the students
to use expert tools.  For example, we required them to write their their
report in \LaTeX, and their presentation slides and analysis description in
plain text with Markdown markup.  We asked them to do all project interaction
using Github tools such as the issue tracker and pull requests.

We set and marked code exercises with git version control and text code files,
rather than interactive alternatives, such as Jupyter Notebooks.

Of course some students complained that they would rather use Facebook chat
for pasting code fragments and Powerpoint for their presentation slides.
Should we have allowed them to do this?  We believe not.  Our own experience
of using the tools we taught is that their true power only becomes apparent
when you learn to use them for all the tasks of analysis, including generating
reports and presentations.  Mixing ``easy'' but heavy tools like Powerpoint
and Facebook with ``simple'' and light tools like text editors and Markdown
causes us to lose concentration as we switch modes from heavy to light and
back again. It is easy to be distracted by the difficulty of getting used to
the light tools, and therefore fail to notice the clarity of thought and
transparency of process that they quickly bring.  Successfully switching from
heavy to light is a process that requires patience and support; it is best
done in the context of a class where there are examples of use from coursework
and support from experienced instructors.

\subsection{What background do students need?}

The requirements for our course were previous classes giving some familiarity
with probability, statistics, and the use of the R programming language.

From our experience of teaching the course, we would be happy to relax the
requirement for probability.  We did not refer to the ideas of
probability in any depth during the course.  Authors JBP and MB have taught
similar material to neuroscience and psychology students who lack training in
probability; the pace of teaching and level of understanding were similar
across the statistics students in this course and the other students we have
taught.

Psychology and neuroscience students do have some statistical training.  We
believe this background is necessary for us to be able to as start quickly as
we did in the analysis of linear regression.

We would also keep the requirement for some programming experience.  Our
course went fairly quickly through basic programming ideas such as for loops,
conditionals, and functions.  We could not have done this for students without
any experience of programming.  In our psychology / neuroscience courses that
used similar material, we required some experience of programming in a
language such as Python, R or Matlab.  It is possible that a brief
introduction would be enough to fulfill this requirement, such as a bootcamp
or multi-day intensive course.

\subsection{Where would such a course fit in a curriculum?}

Our course would not fully qualify a student for independent research in
neuroimaging.  As we discuss above, we did not cover important aspects of
imaging, including spatial and temporal pre-processing of data, random effects
or control of false positives in the presence of multiple comparisons.  For a
full training in imaging, the student would need more training.  Where should
the elements of our course sit in relation to a larger program for training
imagers and other computational scientists?

We think of this course as a foundation.  Students graduating from this course
will be more effective in learning and analysis.  The tools that they have
learned allow the instructor to build analyses from simple blocks, to
show them how algorithms work, and make them simple to extend.  We therefore
suggest that something like this course should be a first course in a
sequence, where the following courses would continue to use these tools for
exposition and for project work.

A full course on brain imaging might start with an introduction to Python and
data analysis, possibly as a week-long bootcamp at the start of the semester.
Something like this course would follow.  There should be follow-up courses
using the same tools for more advanced topics such as machine-learning,
spatial pre-processing, and analysis of other modalities. We suggest that each
of these courses should have group project work as a large component, in which
the student can continue to practice techniques for efficient reproducible
analysis and collaboration.

\subsection{What factors would influence the success of such a course?}

We should note that there were factors in the relative success of this
course that may not apply in other institutions.

Berkeley has as a strong tradition in statistical computing and
reproducibility.  Leo Brieman was a professor in the Berkeley statistics
department, and an early advocate of what we would now call data science.  He
put a heavy emphasis on computational teaching to statisticians.  Sandrine
Dudoit is a professor in the statistics and biostatistics departments, and a
founding core development of the Bioconductor project devoted to reproducible
genomics research in R.  The then head of the statistics department, Philip
Stark, is one of many Berkeley authors (including KJM) to contribute to the
recent book ``The Practice of Reproducible Research``
\citep{kitzes2018practice}. Outside statistics, several of the imaging labs in
Berkeley take this issue seriously and transmit this to their students.  If
there had not been such local interest in the problem of reproducibility, it
would have been more difficult to persuade the students of the importance of
working in a reproducible way, especially given the convenience of less
disciplined working process.

% KJM, FP writings, courses?

Students of statistics and other disciplines at Berkeley are well aware of the
importance of Python in scientific computing, and in industry.  Tech firms
recruit aggressively on campus, and Python is a valuable skill in industry.
The introduction to programming course by the computer science department uses
Python, as does the cross discipline course in Fundamentals of Data Science.
The new Berkeley Institute of Data Science has a strong emphasis towards
Python for scientific computing.

In the same way, a booming tech sector nearby made it more obvious to students
that they would need to learn tools like Git and Github, as these are widely
used in the tech industry.  For example, public activity on Github is one
factor companies use to identify candidates for well-paid positions as
software engineers.

We required students to use \LaTeX to write their final reports.  This was an
easy sell to statistics students, as \LaTeX is widely used in statistics.
Students outside mathematical fields might be less agreeable; if we were
teaching psychology and neuroscience students, we would probably prefer
another plain text markup language, such as Markdown, run through Pandoc
software to write publication quality output.

\section{Conclusion}\label{conclusion}

We conclude with one of the undergraduate student's responses to the question
\emph{``What advice would you give to another student who is considering taking this course?''}:
\begin{quotation}
``[U]nlike most group projects (which last for maybe a few weeks tops or
could conceivably be pulled off by one very dedicated person), this one will
dominate the entire semester. . . . Try to stay organized for the project and
create lots of little goals and checkpoints. You should always be working on
something for the project, whether that's coding, reviewing, writing, etc. Ask
lots of questions and ask them early!''
\end{quotation}

\section*{Conflict of Interest Statement}

The authors declare that the research was conducted in the absence of any
commercial or financial relationships that could be construed as a potential
conflict of interest.

\section*{Author Contributions}

KJM was the lead instructor and was responsible for the syllabus and project timeline;
KJM and MB were responsible for lectures and created homework assignments;
KJM and RB were responsible for labs, readings, quizzes, and grading;
all authors held weekly office hours to assist students with their projects.
KJM and MB wrote the first draft of the manuscript;
all authors wrote sections of the manuscript, contributed to manuscript revision, 
as well as read and approved the submitted version.
