\maketitle

% Curriculum, Instruction, and Pedagogy manuscripts have a 5,000 word limit (~10pages)
\wordcount

\begin{abstract} % 350 words limit
We describe our experience teaching a project-based introduction to
reproducible and collaborative neuroimaging data analysis.

\tiny
 \keyFont{ \section{Keywords:} Neuroimaging, fMRI, Statistics, Reproducible Research, Python, Data Science}
 %All article types: you may provide up to 8 keywords; at least 5 are mandatory.
\end{abstract}

\section{Introduction}

traditional approach vs. our approach to teaching fMRI analysis ...

bottom up (low-level) vs. top down (high-level)

reproducible research \citep{millman2014developing}

Python for neuroimaging analysis \citep{millman2007analysis}

Scientific Python \citep{millman2011python}

interdisciplinary

would it work?

\section{Methods}

During the fall semester of 2015, we co-taught an upper-level undergraduate
(Stat 159) and first-semester graduate (Stat 259) course in the statistics
department at UC Berkeley.\footnote{\url{http://www.jarrodmillman.com/stat159-fall2015/}}

Course purpose

\begin{quote}
A project-based introduction to statistical data analysis. Through case
studies, computer laboratories, and a term project, students learn
practical techniques and tools for producing statistically sound and
appropriate, reproducible, and verifiable computational answers to
scientific questions. Course emphasizes version control, testing,
process automation, code review, and collaborative programming.
Software tools include Bash, Git, Python, and \LaTeX.

\hfill\textit{Reproducible and Collaborative Statistical Data Science (Stat 159/259)}
\end{quote}

our background ... preliminary attempts

student audience

\textbf {Prerequisites:} Statistics 133, Statistics 134, and Statistics 135
(or equivalent). Graduate standing is required to register for Statistics 259.

\textbf {Credit Hours:} 4


\subsection{Core topics}

Our objective for the course was for students to
(1) be proficient at the Unix commandline;
(2) be expert at version control with Git;
(3) be able to write documents in Markdown or \LaTeX (including using pandoc);
(4) be familiar with scientific computing in Python;
(5) understand the computational and statistical issues involved with reproducibility; and
(6) be familiar with computational issues in modern statistical data
analysis through hands-on analysis of functional MRI data.

\subsubsection{Scientific computing with Python}

Using the bash shell;
Introduction to Git;
Introduction to Python

\subsubsection{Statistical analysis of fMRI data}

\subsubsection{Readings}

Annotated reading list?

\begin{itemize}
\item \textbf{Reading 1}: \href{https://osf.io/zqbu2}{L Preeyanon, AB Pyrkosz, and CT Brown.
             ``Reproducible bioinformatics research for biologists.''
Implementing Reproducible Research (2014)}
\item \textbf{Reading 2}: \href{http://arxiv.org/pdf/0906.3662v1}{MA Lindquist. ``The statistical analysis of fMRI data.''}
              %Statistical Science 23.4 (2008)
(2008)
\item \textbf{Reading 3}: \href{http://www.computer.org/cms/Computer.org/ComputingNow/issues/2015/04/T-mcs2011020013.pdf}{F P\'{e}rez, BE Granger, and JD Hunter.
              ``Python: an ecosystem for scientific computing.''}
              %Computing in Science \& Engineering 13.2 (2011)
(2011)
\item \textbf{Reading 4}: \href{http://statweb.stanford.edu/~wavelab/Wavelab_850/wavelab.pdf}{JB Buckheit and DL Donoho.
``Wavelab and reproducible research.'' (1995)}
\item \textbf{Reading 5}: \href{http://www.jarrodmillman.com/publications/millman2014developing.pdf}{KJ Millman and F P\'{e}rez.
              ``Developing open source scientific practice.''}
              %Implementing Reproducible Research (2014)
(2014)
\item \textbf{Reading 6}: \href{http://matthew.dynevor.org/_downloads/does_glm_love.pdf}{JB Poline and M Brett. ``The general linear model and fMRI: does love last forever?'' (2012)}
\item \textbf{Reading 7}: \href{http://journals.plos.org/plosmedicine/article?id=10.1371/journal.pmed.0020124}{JPA Ioannidis. ``Why Most Published Research Findings Are False'' (2005)}
\end{itemize}

\subsubsection{Homeworks and Quizzes}

\subsection{Project}

The majority of the class focuses on a final group project

\begin{quote}
\textbf {\large \\Learning objectives:} Investigate a published fMRI study;
collaborate on a software project; work with complex and large datasets; convolution
(hemodynamic modeling, smoothing); interpolation (slice time correction, image
resampling); optimization (registration, advanced statistics); basic linear
algebra (statistics).
\end{quote}

\begin{quote}
\textbf {\large \\ Project overview:}
The semester project involves ``investigating'' a published result using the
analysis of functional Magnetic Resonance Imaging (MRI).\footnote{Exactly what
each team means by ``investigate'' will need to be defined by each team.  For
example, for one team this might mean a very careful reanalysis of the data
using the original methods as closely as possible.  However, this shouldn't
mean merely running existing scripts used in the original analysis; rather, the
team would need to reimplement the analysis scripts.  Another team might decide
to conduct a different analysis than used in the original study.  For instance,
the published work might use a parametric approach and the team project might
attempt to use a nonparametric technique such as permutation testing.  A third
team might focus on a careful validation of the modeling assumptions made by
the original analysis.}    Functional MRI (fMRI) allows scientists to localize
which parts of the brain are associated with specific cognitive tasks. There is
no expectation that you will have a background in neuroscience. You will learn
everything you need to know about the method in the course. The intention of
focusing on fMRI data is merely to provide a concrete problem domain that
exemplifies the types of programming and statistical challenges present in many
modern statistical applications.  Additionally, the weekly labs will prepare
you---through a series of guided exercises---with the basic skills and
background you need for the group project.

While you will learn the basic methods and tools during lecture and lab, you
will be expected to do additional research and reading in the course of working
on the group project.  For instance, you may need to use a specialized analysis
method or a Python package not covered in the lectures or labs.

As part of your final project grade, you will be required to work on your
project using GitHub's pull request and code review mechanism.  During lecture,
I will cover the exact workflow you will be expected to follow.  Please note
that as this course is explicitly about reproducibility and collaboration your
project grade will not be entirely based on your final report, but will also
reflect how well your group work is reproducible and how effectively you
collaborated using the techniques taught in the course (e.g., pull requests,
code review, testing, etc.)
\end{quote}

\begin{quote}
\begin{flushleft}
Form teams \dotfill Week 5 (Sept. 22)\\
Project proposal \dotfill Week 6 (Oct. 1)\\
Progress presentation \dotfill Week 12 (Nov. 12)\\
Draft report \dotfill Week 12 (Nov. 12)\\
Project presentation \dotfill Week 15 (Dec. 1 \& 3)\\
Final report \dotfill Week 17 (Dec. 14)\\
\end{flushleft}
\end{quote}

\section{Results}

\subsection{Course Projects}



\citep{tom2007neural}

\section{Discussion}

How would this work for Psychology students?

How would this work in a quarter system?

\section{Conclusion}

lessons learned ...


\section*{Conflict of Interest Statement}
%All financial, commercial or other relationships that might be perceived by the academic community as representing a potential conflict of interest must be disclosed. If no such relationship exists, authors will be asked to confirm the following statement: 

The authors declare that the research was conducted in the absence of any commercial or financial relationships that could be construed as a potential conflict of interest.

\section*{Author Contributions}

KJM was the lead instructor and was responsible for the syllabus and project timeline;
KJM and MB were responsible for lectures and created homework assignements;
KJM and RB were responsible for labs, readings, quizzes, and grading;
all authors held weekly office hours to assist students with their projects.
KJM and MB wrote the first draft of the manuscript;
all authors wrote sections of the manuscript, contributed to manuscript revision, 
as well as read and approved the submitted version.

\section*{Funding}

... GSI funding from stat department?
... BIDS funding for writing?
... anything else?

\section*{Acknowledgments}
We would like to thank the following people
... Russ Poldrack and OpenfMRI.org
... Alexander Huth guest lecture
... Stefan, Fernando, Paul, Mark (any one from Practical Neuroimaging?)
... anyone else? 

\bibliographystyle{frontiersinSCNS_ENG_HUMS} % for Science, Engineering and Humanities and Social Sciences articles, for Humanities and Social Sciences articles please include page numbers in the in-text citations
%\bibliographystyle{frontiersinHLTH&FPHY} % for Health, Physics and Mathematics articles
\bibliography{teaching}

%%% Make sure to upload the bib file along with the tex file and PDF
%%% Please see the test.bib file for some examples of references

